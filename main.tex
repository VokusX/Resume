%% If you want to use \orcid or the
%% academicons icons, add "academicons"
%% to the \documentclass options. 
%% Then compile with XeLaTeX or LuaLaTeX.
% \documentclass[10pt,a4paper,academicons]{altacv}
\documentclass[10pt,a4paper]{altacv}

%% AltaCV uses the fontawesome and academicon fonts
%% and packages. 
%% See texdoc.net/pkg/fontawecome and http://texdoc.net/pkg/academicons for full list of symbols.
%% When using the "academicons" option,
%% Compile with LuaLaTeX for best results. If you
%% want to use XeLaTeX, you may need to install
%% Academicons.ttf in your operating system's font %% folder.


% Change the page layout if you need to
\geometry{left=1cm,right=9cm,marginparwidth=6.8cm,marginparsep=1.2cm,top=1cm,bottom=1cm}

% Change the font if you want to.

% If using pdflatex:
\usepackage[utf8]{inputenc}
\usepackage[T1]{fontenc}
\usepackage[default]{lato}

% If using xelatex or lualatex:
% \setmainfont{Lato}

% Change the colours if you want to
\definecolor{VividPurple}{HTML}{2979ff}
\definecolor{SlateGrey}{HTML}{2E2E2E}
\definecolor{LightGrey}{HTML}{666666}
\colorlet{heading}{VividPurple}
\colorlet{accent}{VividPurple}
\colorlet{emphasis}{SlateGrey}
\colorlet{body}{LightGrey}

% Change the bullets for itemize and rating marker
% for \cvskill if you want to
\renewcommand{\itemmarker}{{\small\textbullet}}
\renewcommand{\ratingmarker}{\faCircle}

\begin{document}
\name{David Voicu}
\tagline{Software Engineer}
\title{David Voicu's Resume}
\personalinfo{%
  % Not all of these are required!
  % You can add your own with \printinfo{symbol}{detail}
  \normalsize\email{david@voicu.me}
  \phone{(613) 791-1087}
%  \mailaddress{Address, Street, 00000 County}
%  \location{Sunnyvale, CA}
  \normalsize\homepage{voicu.me}
  \normalsize\github{VokusX}
  \normalsize\linkedin{DavidVoicu}
}

%% Make the header extend all the way to the right, if you want. Extend the right margin by 8cm (=6.8cm marginparwidth + 1.2cm marginparsep)
\begin{adjustwidth}{}{-8cm}
\makecvheader
\end{adjustwidth}

%% Provide the file name containing the sidebar contents as an optional parameter to \cvsection.
%% You can always just use \marginpar{...} if you do
%% not need to align the top of the contents to any
%% \cvsection title in the "main" bar.
\cvsection[page1sidebar]{experience \& projects}

\cvproject
	{SwiftWeather}
    {Personal Project}
    {August 2017 - Present}
\begin{itemize}
	\item {Designed and programmed a weather app in Swift3 to use various APIs to get info.}
    \item {OpenWeatherMap API and Apple’s Core APIs for user location were both used}
    \item {Focus on efficient and optimized code for the best user experience}
\end{itemize}

\divider

\cvevent{Backstage Services}{RBC Bluesfest}{July 2017}{Ottawa, CA}

\begin{itemize}
\item Worked with a team to ensure that the event went on without a hitch
\item Strong focus on customer service and catering to individual requests and needs
\item Worked in conjunction with Catering Services to ensure meals were properly handled, served and enjoyed
\end{itemize}

\divider

\cvproject
	{To the Stars: Mobile Game}
    {Personal Project}
    {October 2016 - June 2017}
\begin{itemize}
	\item {Designed, developed and maintained an Android Game using Unity/C\# and Python.}
    \item {Strong emphasis on full stack development through programming and design.}
    \item {Wrote optimized code to ensure the game would run fast and smooth on a range of devices.}
\end{itemize}

\divider

\cvevent{High School Happenings Producer/Editor}{RogersTV}{September 2016 - June 2017}{Ottawa, CA}

\begin{itemize}
\item Produced, filmed and edited content for a monthly televised broadcast for thousands of viewers.
\item Worked within a team to create content on a strict schedule while following content guidelines.
\item Took initiative within a team in order to effectively collaborate on the project.
\item Emphasis on creativity and collaboration to efficiently produce content. 
\end{itemize}


%% If the NEXT page doesn't start with a \cvsection but you'd
%% still like to add a sidebar, then use this command on THIS
%% page to add it. The optional argument lets you pull up the 
%% sidebar a bit so that it looks aligned with the top of the
%% main column.
% \addnextpagesidebar[-1ex]{page3sidebar}


\end{document}
