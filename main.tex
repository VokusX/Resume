%% If you want to use \orcid or the
%% academicons icons, add "academicons"
%% to the \documentclass options. 
%% Then compile with XeLaTeX or LuaLaTeX.
% \documentclass[10pt,a4paper,academicons]{altacv}
\documentclass[10pt,a4paper]{altacv}

%% AltaCV uses the fontawesome and academicon fonts
%% and packages. 
%% See texdoc.net/pkg/fontawecome and http://texdoc.net/pkg/academicons for full list of symbols.
%% When using the "academicons" option,
%% Compile with LuaLaTeX for best results. If you
%% want to use XeLaTeX, you may need to install
%% Academicons.ttf in your operating system's font %% folder.


% Change the page layout if you need to
\geometry{left=1cm,right=9cm,marginparwidth=6.8cm,marginparsep=1.2cm,top=1cm,bottom=1cm}

% Change the font if you want to.

% If using pdflatex:
\usepackage[utf8]{inputenc}
\usepackage[T1]{fontenc}
\usepackage[default]{lato}

% If using xelatex or lualatex:
% \setmainfont{Lato}

% Change the colours if you want to
\definecolor{VividPurple}{HTML}{2979ff}
\definecolor{SlateGrey}{HTML}{2E2E2E}
\definecolor{LightGrey}{HTML}{666666}
\colorlet{heading}{VividPurple}
\colorlet{accent}{VividPurple}
\colorlet{emphasis}{SlateGrey}
\colorlet{body}{LightGrey}

% Change the bullets for itemize and rating marker
% for \cvskill if you want to
\renewcommand{\itemmarker}{{\small\textbullet}}
\renewcommand{\ratingmarker}{\faCircle}

\begin{document}
\name{David Voicu}
\tagline{SOFTWARE ENGINEER | TECH ENTHUSIAST | CODE NINJA}
\title{David Voicu's Resume}
\personalinfo{%
  % Not all of these are required!
  % You can add your own with \printinfo{symbol}{detail}
  \normalsize\email{david@voicu.me}
  \phone{(613) 791-1087}
%  \mailaddress{Address, Street, 00000 County}
%  \location{Sunnyvale, CA}
  \normalsize\homepage{voicu.me}
  \normalsize\github{VokusX}
  \normalsize\linkedin{DavidVoicu}
}

%% Make the header extend all the way to the right, if you want. Extend the right margin by 8cm (=6.8cm marginparwidth + 1.2cm marginparsep)
\begin{adjustwidth}{}{-8cm}
\makecvheader
\end{adjustwidth}

%% Provide the file name containing the sidebar contents as an optional parameter to \cvsection.
%% You can always just use \marginpar{...} if you do
%% not need to align the top of the contents to any
%% \cvsection title in the "main" bar.
\cvsection[page1sidebar]{experience \& projects}

\cvproject
	{weCU - Particle Electron}
    {Hackathon Project}
    {February 2018}
\begin{itemize}
	\item {A Particle Electron Smart Home monitoring system without the need for power or wifi by utilizing the Telus GSM network.}
	\item {Wrote backend interface for the hardware and the Firebase server, optimized networking for low power use while being secure.}
	\item {Helped develop a front end using React Native to convert realtime JSON data from a Firebase server for display on a user friendly front end.}
	\item {Code can be found on Devpost: https://devpost.com/software/we-cu}
\end{itemize}

\divider

\cvproject
	{voicu.me}
    {Personal Website}
    {December 2017 - Present}
\begin{itemize}
	\item {Used a combination of HTML, CSS, and JS along with frameworks such as jQuery to create a website portfolio hosted on AWS.}
    \item {Focused on making an aesthetically pleasing, yet intuitive design for the user with a focus on Material Design.}
    \item {Designed a responsive system to ensure compatibility over a multiple range of devices and displays while being optimized and lightweight.}
\end{itemize}

\divider

\cvproject
	{SwiftWeather}
    {Mobile App}
    {August 2017 - October 2016}
\begin{itemize}
	\item {Designed and programmed a weather app in Swift3 to report real time weather data and forecasts to users.}
    \item {Took advantage of the OpenWeatherMap API and Apple's CoreKit in order to handle weather and location data.}
    \item {Focus on efficient and optimized code for the best user experience to ensure that the app ran quickly and smoothly.}
\end{itemize}

\divider

\cvproject
	{To the Stars: Mobile Game}
    {Mobile Game}
    {October 2016 - June 2017}
\begin{itemize}
	\item {Designed, developed and maintained an Android Game using Unity/C\# and Python.}
    \item {Strong emphasis on full stack development through programming and design.}
    \item {Wrote optimized code to ensure the game would run quickly and smoothly on a range of devices.}
\end{itemize}

\divider


%% If the NEXT page doesn't start with a \cvsection but you'd
%% still like to add a sidebar, then use this command on THIS
%% page to add it. The optional argument lets you pull up the 
%% sidebar a bit so that it looks aligned with the top of the
%% main column.
% \addnextpagesidebar[-1ex]{page3sidebar}


\end{document}
